\documentclass[notitlepage]{article}
\usepackage{bibunits}
\usepackage{comment}
\usepackage{graphicx}
\usepackage{amsmath}
\usepackage{datetime}
\usepackage{numprint}
\usepackage{palatino}
\usepackage{url}
\usepackage{footmisc}
\usepackage{endnotes}
\usepackage{listings}

\begin{document}

\nplpadding{2}
\newdateformat{isodate}{
	\THEYEAR -\numprint{\THEMONTH}-\numprint{\THEDAY}}

\title{Project Freedom Executive Summary}
\author{Gabriel Aldous}
\date{\isodate\today}

\maketitle

\tableofcontents

\newpage
\section{Project Overview}

Project Freedom is designed to

\section{Mathematical Principles}

The fundamental operation which this program seeks to
foster a greater understanding of is called the Fourier
Transform, or more specifically the Fast Fourier Transform (FFT).
This technique is often used in signal processing applications,
as is the case in this project. The FFT is utilized to turn some
signal data (such as an image or an audio file) into a different domain
called the frequency domain. Simply speaking, the FFT takes the input and
restructures it so that information about which frequencies are more prevalent
is easily accessible.

The FFT has applications in many areas, notably in image compression formats
like JPEG and MP3, filtering operations, wireless communication, and much more.
For more details about how the FFT works, and the linear algebra supporting it,
view the (LONG FORM DOCUMENT).

\section{Results}

\end{document}