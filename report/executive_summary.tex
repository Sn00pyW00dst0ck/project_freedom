\documentclass[notitlepage]{article}
\usepackage{bibunits}
\usepackage{comment}
\usepackage{graphicx}
\usepackage{amsmath}
\usepackage{datetime}
\usepackage{numprint}
\usepackage{palatino}
\usepackage{url}
\usepackage{footmisc}
\usepackage{endnotes}
\usepackage{listings}
\usepackage[colorlinks=true, urlcolor=blue, linkcolor=red]{hyperref}

\begin{document}

\newdateformat{isodate}{
	\THEYEAR -\numprint{\THEMONTH}-\numprint{\THEDAY}}

\begin{center}
	Project Freedom Executive Summary
	\\
	Gabriel Aldous
	\\
	MAS4115
	\\
	\href{https://github.com/Sn00pyW00dst0ck/project_freedom}{Project on GitHub}
\end{center}
\textbf{Project Overview}

Project Freedom is a easy to use CLI program that
users can interact with in order to perform their own experiments
with the Fast Fourier Transform and related operations on both images and audio files.
Basic filtering operations are supported for images and audio, and
there are numerous ways to visualize how the files have changed
and preview those changes. Additionally, the application supports
easy modification of the file system, and the classes employed
to perform these image modifications can easily be repurposed by
other programmers to add additional functionalities.

In addition to the basic manipulations, there is an implemented
advanced 'hybridization' technique for both audio and image files.
Image files can be 'hybridized' so that the image appears to be one
thing up close, and another when viewed from far away, while audio
files can be modified to make it appear that one sound is being heard
within a different environment.

This project is implemented in Python, and supports building on
Python versions 3.9-3.12 (though it was only tested on version 3.12).
It relies on Numpy, Matplotlib, Scipy, and PyGame, and instructions for
setting up and running the program are available in the README file.
To view the results of various operations performed by Project Freedom,
see the github repository.
\\\\
\textbf{Mathematical Principles}

The fundamental operation which this program seeks to
foster a greater understanding of is called the Fourier
Transform, or more specifically the Fast Fourier Transform (FFT).
This technique is often used in signal processing applications,
as is the case in this project. The FFT is utilized to turn some
signal data (such as an image or an audio file) into a different domain
called the frequency domain. Simply speaking, the FFT takes the input and
restructures it so that information about which frequencies are more prevalent
is easily accessible.

In the context of the custom hybridizations, audio hybridization is performed using
a convolution, while image hybridization is performed by filtering the high detail portions
of one image (the edges \& lines) and overlaying it on the low detail portion of another
image (the general colors \& shapes). This results in an image which can trick human perception
and appear differently from different viewing distances.

The FFT has applications in many areas, notably in image compression formats
like JPEG and MP3, filtering operations, wireless communication, and much more.
For more details about how the FFT works, and the linear algebra supporting it,
view the \href{}{}.

\end{document}